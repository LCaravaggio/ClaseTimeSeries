\documentclass{beamer}
\usetheme{Boadilla}
\usepackage{amsmath} 
\usepackage{ragged2e} 
\usepackage{hyperref}

\title{Introducción a Series de Tiempo}
\author{Leonardo A. Caravaggio}
\date{Abril 2022}

\begin{document}

\maketitle

\begin{frame}
\frametitle{¿Qué es una serie de tiempo?}

\justify
Una serie de tiempo es un conjunto de datos indexado en el tiempo.\\
\vspace{5mm} %5mm vertical space
Constituyen la principal fuente de información utilizada en la estimación de Modelos Econométricos. 
\end{frame}


\begin{frame}
\frametitle{Componentes}

\justify
La relación entre estas variables puede escribirse de la siguiente manera: \\
\vspace{5mm} %5mm vertical space
\begin{center}
Observación=Componente sistémica o determinística + Perturbación aleatoria \\
\end{center}
\vspace{5mm} %5mm vertical space
Es decir: 
\[
y_t=x_{kt}+u_t
\]
donde $u_t$ representa un término de error, es decir, una diferencia aleatoria entre las variables. 

\end{frame}


\begin{frame}
\frametitle{Componentes}

\justify
El procesamiento tradicional para la modelización de las series de tiempo consiste en la descomposición, sea en forma aditiva o multiplicativa, de sus componentes elementales: Tendencia (T), Variación Estacional (S), Variación Cíclica (C) y componente aleatoria (e). Si se adopta el esquema aditivo, la representación sería: 
\[
Y_t=T_t+S_t+C_t+u_t
\]
\end{frame}

\begin{frame}
\frametitle{Estacionariedad}
\justify
No confundir \textbf{estacionariedad} con estacionalidad\\
\vspace{5mm} %5mm vertical space
Un proceso estocástico es \textbf{estacionario en sentido estricto} si todas las variables aleatorias que componen el proceso están idénticamente distribuidas, independientemente del momento del tiempo en que se estudie el proceso.\\
\vspace{5mm} %5mm vertical space
La \textbf{estacionariedad fuerte} es de difícil constatación empírica, por lo que en general se utiliza el criterio de estacionariedad débil. Se dice que un proceso es débilmente estacionario si se verifican simultáneamente las condiciones de estacionariedad en media y en varianza.

\end{frame}

\begin{frame}
\frametitle{Práctica}
\justify
Dejemos por un rato la teoría para ver algo de práctica. \\
\vspace{5mm} %5mm vertical space
Vamos a bajar algunas series de tiempo y observarlas usando Google Colab y la API de Alphacast. \\
\vspace{5mm} %5mm vertical space
Ver Ejercicio 1 en el Repositorio
\end{frame}


\begin{frame}
\frametitle{Test de Estacionariedad}
\justify
El test más conocido para chequear si una serie es no estacionaria (es decir, si tiene raíz unitaria) es el \href{https://en.wikipedia.org/wiki/Augmented_Dickey-Fuller_test}{Test de Dicky Fuller aumentado}.\\

\vspace{5mm} %5mm vertical space
Existen alternativas, como por ejemplo el test de Phillips–Perron (PP) o el test de ADF-GLS. 
\end{frame}


\begin{frame}
\frametitle{Análisis Econométrico de las Series de Tiempo}
\justify
La mayoría de las series empíricas pertenecientes al ámbito de la economía tienen un comportamiento \textbf{no estacionario}.\\
\vspace{5mm} %5mm vertical space
Esto quiere decir, que para evitar modelizar una correlación espuria, será necesario tornarlos estacionarios. Esto puede hacerse calculando el operador diferencia, aplicando una escala logarítmica, una escala inversa, etc. 
\end{frame}


\begin{frame}
\frametitle{Modelo sencillo}
\justify
Podemos pensar un modelo sencillo se estimación del EMAE en base a dos estimadores adelantados de la actividad, el IPI y el ISAC.\\
\vspace{5mm} %5mm vertical space
\[
EMAE_t=\beta_0+\beta_1IPI_t+\beta_2ISAC_t+u_t
\]
\end{frame}

\begin{frame}

\frametitle{Más práctica}
Podemos probar ahora los test de estacionariedad usando el paquete Statsmodel, y armar el modelo sencillo bajando los datos con la API de Alphacast.\\ 

\vspace{5mm} %5mm vertical space
Ver Ejercicio 2 en el repositorio.\\
\end{frame}


\begin{frame}
\frametitle{Modelos VAR}
Un problema del modelo sencillo planteado anteriormente es que le imponen una dirección de causalidad a las variables.\\ 
\vspace{5mm} %5mm vertical space
Para evitar esto, cuando dos series de tiempo están interrelacionadas, se puede usar un modelo de Vectores Autoregresivos (VAR).\\
\end{frame}

\begin{frame}
\frametitle{Modelos VAR}
El nombre del modelo es bastante intuitivo respecto de su estructura. Se trata de un modelo autoregresivo, pero multivariado, permitiendo así la interacción entre las variables y los rezagos de las mismas.\\ 
\vspace{5mm} %5mm vertical space
\[
Y_{1t}=\alpha_1+\beta_{11}Y_{1,t-1}+\beta_{12}Y_{2,t-1}+u_1t
\]
\[
Y_{2t}=\alpha_2+\beta_{21}Y_{1,t-1}+\beta_{22}Y_{2,t-1}+u_2t
\]
\\
\vspace{5mm} %5mm vertical space
Este sería un modelo VAR(1) porque cada variable tiene un rezago. 

\end{frame}


\end{document}
